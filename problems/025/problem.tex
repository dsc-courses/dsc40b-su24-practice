%% source: 2021-fa-midterm_01
%% tags: [asymptotic notation]
\begin{prob}
    Let $f$ be the piecewise function defined as:
    \[
        f(n) = \begin{cases}
            1, & \text{if $n$ is a multiple of 1 million},\\
            n^2, &\text{otherwise}.
        \end{cases}
    \]

    If you were to plot $f$, the function would look like $n^2$, but would have point
    discontinuities where it ``jumps'' back to 1 whenever $n$ is a multiple of
    1 million.

    True or False: $f(n) = \Theta(n^2)$.

    \tF{}

    \begin{soln}
        False.

        $f(n)$ \textit{is} upper bounded by $O(n^2)$, but it doesn't have a
        lower bound of $\Omega(n^2)$, which is necessary for it to be
        $\Theta(n^2)$.

        To see why, remember that for $f(n)$ to be $\Omega(n^2)$, there must be
        a positive constant $c$ so that $f(n)$ stays above $c\cdot n^2$ for all
        $n$ greater than some $n_0$, meaning that it goes above $c n^2$
        \textit{and it stays above $c n^2$} forever, after some point.

        Pick whatever positive $c$ you'd like -- the function $c\cdot n^2$
        grows larger and larger as $n$ increases, and eventually becomes larger
        than 1. But the function $f(n)$ keeps dipping down to 1, meaning that
        it doesn't stay above $c\cdot n^2$ for all $n$ when $n$ is large.
        Therefore, $f(n)$ is not $\Omega(n^2)$, and therefore also not
        $\Theta(n^2)$.
    \end{soln}

\end{prob}
