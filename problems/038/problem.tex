%% source: 2021-fa-midterm_01
%% tags: [theoretical lower bounds]
\begin{prob}
    Suppose you are given a (possibly unsorted) array of $n$ floating point numbers and are tasked with counting the number
    of elements in the array which are within ten of the array's maximum (you are not told the maximum).
    That is, you must count the number of elements $x$ such that $m - x \geq 10$, where $m$ is the array's
    maximum. You may assume that the array does not contain duplicate numbers.

    Now you may assume that the array is \textbf{sorted}.

    This question has two parts:

    First, give the tight theoretical lower bound for the worst case
    time complexity of any algorithm which solves this problem:

    \begin{soln}
        $\Theta(\log n)$
    \end{soln}

    Second, give a short description of an optimal algorithm.
    Your description does not need to be exact or very long (3 to 4 sentences will do).
    You do not \textit{need} to provide pseudocode, but you can if you wish.

    \begin{soln}
        The maximum $m$ can be found in $\Theta(1)$ time -- it is the last element of the array.
        Use binary search ($\Theta(\log n)$ time) to look for $m - 10$, but
        modify the code so that the last-checked index is returned if $m -
        10$ is not found (everything to the right of that index is at least
        $m - 10$). Subtract this index from $n$ to find the number of
        elements within 10 of the maximum.
    \end{soln}
\end{prob}
