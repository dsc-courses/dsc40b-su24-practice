%% source: 2021-fa-midterm_01
%% tags: [binary search]
\begin{prob}
    Consider the problem of searching for an element \mintinline{python}{t} in a
    \textit{sorted} array, with the added requirement that if \mintinline{python}{t} is in
    the array multiple times, we return the index of the \textit{first} occurrence.
    For example, if \mintinline{python}{arr = [3, 4, 4, 4, 5, 8]}, and \mintinline{python}{t = 4}, the code
    should return \mintinline{python}{1}, since the first \mintinline{python}{4} is at index one.

    \mintinline{python}{binary_search} as implemented in lecture may not necessarily return
    the index of the first occurrence of \mintinline{python}{t}, but it can be modified
    so that it does.

    Fill in the blanks below so that the function returns the index of the
    first occurence of \mintinline{python}{t}. Hint: \mintinline{python}{last_seen} should keep track of the
    index of the last known occurrence of \mintinline{python}{t}.

    \begin{minted}{python}
        import math
        def mbs(arr, t, start, stop, last_seen=None):
            if start - stop <= 0:
                return last_seen

            middle = math.floor((stop - start) / 2)

            if arr[middle] == t:
                # _________ <-- fill this in
            elif arr[middle] < t:
                # _________ <-- fill this in
            else:
                # _________ <-- fill this in
    \end{minted}

    \begin{soln}

        Here is how the blanks should be filled in:

        \begin{minted}[autogobble]{python}
            if arr[middle] == t:
                return mbs(arr, t, start, middle, middle)
            elif arr[middle] < t:
                return mbs(arr, t, middle + 1, stop, last_seen)
            else:
                return mbs(arr, t, start, middle, last_seen)
        \end{minted}

        If we see \mintinline{python}{arr[middle] == t}, we know that this is either
        the first occurrence of \mintinline{python}{t} in the array, or the 
        first occurrence of \mintinline{python}{t} in the left half of the array.
        Therefore, we make a recursive call to \mintinline{python}{mbs} on the left
        half of the array, keeping track of \mintinline{python}{middle} as the index
        of the leftmost occurrence of \mintinline{python}{t} we have seen so far (this is
        done by passing it as the value to \mintinline{python}{last_seen}).

        The other recursive calls are the same as in the original \mintinline{python}{binary_search}
        function, except that we pass \mintinline{python}{last_seen} as an argument
        to the recursive calls. This is because we want to keep track of the leftmost
        occurrence of \mintinline{python}{t} we have seen so far.

        Eventually, we will reach a point where \mintinline{python}{start - stop <= 0}.
        At this point, we return \mintinline{python}{last_seen}, which is the index
        of the leftmost occurrence of \mintinline{python}{t} we have seen so far.
        This will necessarily be the leftmost occurrence of \mintinline{python}{t}
        in the array.

    \end{soln}

\end{prob}
