%% source: 2023-sp-redemption_midterm_01
%% tags: [mergesort, recursion]
\begin{prob}

    Consider the code for \mintinline{python}{mergesort} and \mintinline{python}{merge} where a new
    \mintinline{python}{print}-statement has been added to \mintinline{python}{merge}:

    \begin{minted}{python}
        def mergesort(arr):
            """Sort array in-place."""
            if len(arr) > 1:
                middle = math.floor(len(arr) / 2)
                left = arr[:middle]
                right = arr[middle:]
                mergesort(left)
                mergesort(right)
                merge(left, right, arr)

        def merge(left, right, out):
            """Merge sorted arrays, store in out."""
            left.append(float('inf'))
            right.append(float('inf'))
            left_ix = 0
            right_ix = 0

            for ix in range(len(out)):
                print("Here!") # <---- the newly-added line
                if left[left_ix] < right[right_ix]:
                    out[ix] = left[left_ix]
                    left_ix += 1
                else:
                    out[ix] = right[right_ix]
                    right_ix += 1
    \end{minted}

    Suppose \mintinline{python}{mergesort} is called on an array of size $n$. In total, how
    many times will \mintinline{python}{"Here!"} be printed?

    \begin{choices}
        \choice $\Theta(\log n)$
        \choice $\Theta(n)$
        \correctchoice $\Theta(n \log n)$
        \choice $\Theta(n^2)$
    \end{choices}

    \begin{soln}
        $\Theta(n \log n)$
    \end{soln}

\end{prob}
