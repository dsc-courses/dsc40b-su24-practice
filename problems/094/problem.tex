%% source: 2023-sp-redemption_midterm_01
%% tags: [time complexity]
\begin{prob}
    Suppose \mintinline{python}{bar} and \mintinline{python}{baz} are two functions.
    Suppose \mintinline{python}{bar}'s time complexity is $\Theta(n^2)$, while \mintinline{python}{baz}'s
    time complexity is $\Theta(n)$.

    Suppose \mintinline{python}{foo} is defined as below:

    \begin{minted}{python}
        def foo(n):
            # will be True if n is even, False otherwise
            is_even = (n % 2) == 0
            if is_even:
                bar(n)
            else:
                baz(n)
    \end{minted}

    Let $T(n)$ be the time taken by \mintinline{python}{foo} on an input of sized $n$.
    True or False: $T(n) = \Theta(n^2)$.

    \tF{}

    \begin{soln}
        False.

        This function is not $\Theta(n^2)$. For that matter, it is also not $\Theta(n)$.
        It \textit{is} $O(n^2)$ and $\Omega(n)$, though.

        This function cannot be $\Theta(n^2)$ because there are no positive constants
        $c, n_0$ such that $T(n) > c n^2$ for all $n > n_0$. You can see this by imagining
        the plot of the time taken by \mintinline{python}{foo} as a function of $n$.
        It "oscillates" between something that grows like $n$ and something that grows
        like $n^2$. If you tried to lower bound it with $cn^2$, $T(n)$ would eventually
        dip below $cn^2$, since $cn^2$ grows faster than $n$.
    \end{soln}
\end{prob}
