%% source: 2023-sp-redemption_midterm_01
%% tags: [theoretical lower bounds]
\begin{prob}
    Consider the following problem: given a (possibly unsorted) list of size
    $n$ containing only ones and zeros, determine if the number of ones is
    equal to the number of zeros.

    \begin{subprobset}

        \begin{subprob}
            What is a \textbf{tight} theoretical lower bound for this problem?
            State your answer as a function of $n$ using asymptotic notation.

            \begin{soln}
                $\Theta(n)$
            \end{soln}
        \end{subprob}

        \begin{subprob}
            Now assume that the input array is guaranteed to have the following structure:
            there will be $a$ ones, followed by $b$ zeros, followed again by $a$ ones.
            An example of such an array is:
                \mintinline{python}{[1, 1, 0, 0, 0, 1, 1]}.
            In this example, $a = 2$ and $b = 3$.

            Consider the same problem as before: that is, given an array
            satisfying the above assumption, determine if the number of ones in
            the array is equal to the number of zeros. Of course, $a$ and $b$
            are not known ahead of time.

            What is a \textbf{tight} theoretical lower bound for this problem?

            \begin{soln}
                $\Theta(1)$

                Let $n$ be the length of the array.
                Assume that $n$ is even (if it is odd, we can immediately say that
                there are not an equal number of ones and zeros). In fact, we can assume
                that $n$ is divisible by four, since $n = a + b + a = 2a + b$, which
                must equal $4a$ if there are an equal number of ones and zeros.

                For there to be an equal number of ones as zeros, the middle
                $n/2$ elements must be zero, while the first $n/4$ and last
                $n/4$ must be ones. If that is the case, then the last one in
                the first group of ones will occur at index $n / 4 - 1$, and
                the first zero will occur at $n / 4 $

                A constant-time algorithm is to check the element at index $n/4
                - 1$ and verify that it is a one. Then we check the element at
                index $n/4 $ and check that it is a zero.

                Consider, for example, this array with $n = 10$:
                \[
                    \mintinline{python}{[1, 1, 1, 0, 0, 0, 0, 1, 1, 1]}
                \]
                Then $n / 10 = 10 / 4 = 2.5 = 2$. Therefore, we check the
                elements at index 1 and 2 in constant time and find them to
                both be one, which tells us that this array does not have an
                equal number of zeros and ones.

                Now consider the array below:
                \[
                    \mintinline{python}{[1, 1, 0, 0, 0, 0, 1, 1]}
                \]
                Since $n = 8$, we again check the elements at index 1 and 2. We find 1
                and 0, which tells us that the array \textit{does} have an equal number
                of zeros and ones, as expected.
            \end{soln}

        \end{subprob}

    \end{subprobset}

\end{prob}
